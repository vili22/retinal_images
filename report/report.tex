\documentclass[aps,prb,10pt,twocolumn,groupedaddress]{revtex4-1}
%\setlength\topmargin{4.6mm}
%\documentclass{[prl,twocolumn]{revtex4-1}
%\usepackage[ansinew]{inputenc}
%\usepackage[latin1]{inputenc}% PROPER ENCODINGS
%\usepackage[T1]{fontenc}%      FOR FINNISH TEXT
%\usepackage[finnish]{babel}% 
%\usepackage[official]{eurosym}

%\usepackage{subfig}
\usepackage{caption}
\usepackage{subcaption}
\usepackage{graphicx}
\usepackage{epsfig}
\usepackage{epstopdf}
\usepackage{amsmath}
\usepackage{blkarray}
\usepackage{multirow}
\usepackage{mathtools}
\usepackage[font=small,labelfont=bf]{caption}

%\usepackage{subfig}
%\usepackage[footnotesize]{caption}
%\pagestyle{empty}
%\setlength{\textwidth}{140mm}
%\setlength{\textheight}{240mm}
%\setlength{\parindent}{0mm}
%\setlength{\parskip}{3mm plus0.5mm minus0.5mm}
\bibliographystyle{apsrev4-1}
%%%%%%%%%%%%%%%%%%%%%%%%%%%%%%%%%%
\begin{document}

\title{Automatic Detection of Blood Vessels From Retinal Images Using Convolutional Neural Network }
\date{\today}
\author{Ville Virkkala, Jarno Leppänen}
\affiliation{Ekahau Research}

\begin{abstract}
  A genre of a song can be estimated based on its music signal's
  characteristics . In this work we use two classifiers, Bayes Classifier
  and Logistic classifier, to classify songs into one of ten possible genres. The
  two classifiers are trained against the training data and their performance is
  compared against each other. In this work we show that both classifiers
  perform much better compared to random guess. However their capability to
  classify all songs is clearly limited the accuracy for both classifiers being
  around 60\%.
\end{abstract}

\maketitle

\section{Introduction}

Several studies have used neural networks in automatic blood vessel detection from retinal images. These methods can be roughly divided into two groups: patch based segmentation of blood vessels and methods based on fully convolutional neural networks. In patch based image segmentation the image is traversed through pixel by pixel. For each pixel a patch of fixed size, centered at the pixel, is taken from the image and fed to neural network that classifies the pixel into certain class.
The advantages of patch based method are their simplicity and ease of training. However, their main disadvantage is the high computational load when doing inference, because for each pixel separate patch is taken that is fed to neural network. Fully convolutional neural network are the current state of the art method in image segmentation. The huge advantage of fully convolutional neural network is the huge speed up compared to patch based methods, because the whole image is fed only once as whole to the neural network. In addition there is more contextual information available in fully convolutional neural networks because the whole image is processed at once instead of using smaller patches. 

In this work patch based semantic segmentation method using convolutional neural network is developed to detect blood vessels from retinal images. The performance of the developed method is validated against several test images.
The paper is organized as follows. The used data-set and the computational
methods are described in detail in Sec. \ref{sec:methods}. In Sec.
\ref{sec:results} the results for the both logistic regression- and
Bayes-classifier are given. Sec. \ref{sec:conclusions} is a
summary of the results and the differences between the two classifiers are
discussed.

\section{Used data-set and computational methods}
\label{sec:methods}
\subsection{Used data-set}
\label{sec:used_data_set}
The data-set consisted of 4363 songs and was divided into training and test data
sets including every third song to test set and rest of the songs to training
set. Each song contained 264 features and the songs were labeled to 10
different categories. The gatecories were: 1 Pop Rock, 2 Electronic, 3 Rap,
4 jazz, 5 Latin, 6 RnB, 7 International, 8 Country, 9 Reggae and 10 Blues.
The musical characteristics of the songs were packed to a feature
vector of length 256. The first 48 elements in the feature vector can be
associated to timbre, the next 48 elements to pitch and the final 168 features
to rhythm. The distribution of the features resembled in most cases a Gaussian
distribution or a skew symmetric distribution. This is illustrated figures
\ref{fig:feature_distribution}a and \ref{fig:feature_distribution}b.


\subsection{Computational methods}
\label{sec:computational_methods}
In this work two different methods were used to classify the songs to different
genres. First method is logistic-regression method in which the logistic-loss
is minimized iteratively using the gradient descent method.
The other method used is the Bayes-classifier which classifies the song to
certain category that gives the maximum posterior probability with respect to
label $i$. Both methods are described below in detail. In addition we studied
the effect of feature extraction and for that purpose we used principal
component analysis method to exclude features with little impact. 
\subsubsection{Logistic-regression}

\subsubsection{Principal component analysis}

\begin{center}
  \begin{table}
    \caption{Confusion matrix corresponding to classification obtained using
      logistic regression. The column direction indicates the true value and
      the row direction is the predicted value.  The labels $1\ldots10$ are
      the ten music genres specified in section \ref{sec:used_data_set}.}
    \begin{tabular*}{0.45\textwidth}{@{\extracolsep{\fill}}c|cccccccccc}
        & 1 & 2 & 3 & 4 & 5 & 6 & 7 & 8 & 9 & 10\\
      \hline
      1 & 652 & 35 & 7 & 9 & 4 & 10 & 1 & 3 & 2 & 3\\
      2 & 57 & 130 & 9 & 4 & 2 & 1 & 0 & 2 & 2 & 0\\
      3 & 14 & 9 & 82 & 4 & 2 & 1 & 0 & 0 & 1 & 0 \\
      4 & 25 & 3 & 0 & 43 & 0 & 2 & 0 & 1 & 0 & 2 \\
      5 & 38 & 4 & 1 & 5 & 11 & 1 & 1 & 0 & 3 & 0 \\
      6 & 37 & 6 & 12 & 8 & 4 & 25 & 0 & 0 & 0 & 0 \\
      7 & 34 & 6 & 1 & 2 & 2 & 4 & 3 & 1 & 0 & 0 \\
      8 & 50 & 0 & 0 & 1 & 2 & 1 & 0 & 9 & 0 & 0 \\
      9 & 2 & 5 & 0 & 2 & 1 & 0 & 0 & 8 & 0 & 27 \\
      10 & 21 & 0 & 0 & 6 & 1 & 2 & 0 & 1 & 0 & 3\\
      \end{tabular*}
    \label{tab:confusion_log_reg}
  \end{table}
\end{center}


\section{Conclusions}
\label{sec:conclusions}
In this work we used logistic-regression and Bayes-classifier to classify songs
to different genres based on the music signal's characteristics. For logistic
regression the obtained accuracy for test set was 0.67 and logistic-loss 0.27.
For the external data set used the obtained accuracy and logistic-loss were 0.65
and 0.178 respectively in the case of logistic-regression. For the
Bayes-classifier the obtained accuracy and logistic-loss were 0.53 and 0.33 for
the test-data and for external data set 0.32 and 1.17 respectively. According to
obtained results both classifiers performed clearly better than random guess,
but remained far from perfect classification. From the two classifiers used
the logistic-regression classifier performed clearly better. The logistic
classifier also generalized much better to completely new data giving nearly
equal performance for test data set and external data set.

\bibliography{references}

\end{document} 






