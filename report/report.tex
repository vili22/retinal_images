\documentclass[aps,prb,10pt,twocolumn,groupedaddress]{revtex4-1}
%\setlength\topmargin{4.6mm}
%\documentclass{[prl,twocolumn]{revtex4-1}
%\usepackage[ansinew]{inputenc}
%\usepackage[latin1]{inputenc}% PROPER ENCODINGS
%\usepackage[T1]{fontenc}%      FOR FINNISH TEXT
%\usepackage[finnish]{babel}% 
%\usepackage[official]{eurosym}

%\usepackage{subfig}
\usepackage{caption}
\usepackage{subcaption}
\usepackage{graphicx}
\usepackage{epsfig}
\usepackage{epstopdf}
\usepackage{amsmath}
\usepackage{blkarray}
\usepackage{multirow}
\usepackage{mathtools}
\usepackage[font=small,labelfont=bf]{caption}

%\usepackage{subfig}
%\usepackage[footnotesize]{caption}
%\pagestyle{empty}
%\setlength{\textwidth}{140mm}
%\setlength{\textheight}{240mm}
%\setlength{\parindent}{0mm}
%\setlength{\parskip}{3mm plus0.5mm minus0.5mm}
\bibliographystyle{apsrev4-1}
%%%%%%%%%%%%%%%%%%%%%%%%%%%%%%%%%%
\begin{document}

\title{Logistic regression and Bayes-classifier study of classification of
  songs to genres based on timbre, pitch and rhythm of the music signal }
\date{\today}
\author{Ville Virkkala, Jarno Leppänen}
\affiliation{Ekahau Research}

\begin{abstract}
  A genre of a song can be estimated based on its music signal's
  characteristics . In this work we use two classifiers, Bayes Classifier
  and Logistic classifier, to classify songs into one of ten possible genres. The
  two classifiers are trained against the training data and their performance is
  compared against each other. In this work we show that both classifiers
  perform much better compared to random guess. However their capability to
  classify all songs is clearly limited the accuracy for both classifiers being
  around 60\%.
\end{abstract}

\maketitle

\section{Introduction}
An automatic music transcription, \textit{i.e.}, notating a piece of music to
a speficic genre, \textit{e.g.}, Blues, dates back to 1970s when first attempts
towards automatic music transcription were made\cite{musictranscription}. Since
then interest in
automatic transcription of music has grown rapidly and various approaches,
statistical methods, modelling human auditory system, have been applied to
music transcription problem. However even today an expert human musician often
beats a state-of-the-art automatic transcription system in accuracy.

Characteristics of music signal that are useful in classification of a song
are \textit{timbre}, \textit{rhythm}, \textit{pitch}, \textit{loudness} and
\textit{duration}\cite{musictranscription} from which the three first one,
described below are used in this work.
\begin{itemize}  
\item The timbre of the music can be most easily described as the factor
  which separates two sources of music from each other. For example if the same
  song is played by violin or a guitar the timbre is called the character which
  separates the violin from the guitar.
\item The pitch is related to frequency scale of a song a can be defined as the
  frequency of the sine-wave fitted to target sound by human listener.
\item The rhythm of the music can be described as arrangement of sounds as time
  flows.
\end{itemize}

In classification problem the object is classified into a certain class based
on it's characteristics called features. A linear classifier does the
classification by making a linear combination of the features and converting
the resulting value into a class or a probability that the object belongs to
given class. In logistic regression the feature vector of the object is
transformed into a probality by taking a linear combination of features and
mapping the result into interval $\lbrack 0, 1\rbrack$ using a sigmoid function.
The Bayes-classifier in contrast assumes that the feature vector is drawn from a
multidimensional-Gaussian distribution. The posterior probability of the object
belonging to a certain class is then obtained as a product of the prior of the
class and the probability to sample the given feature vector from the
multidimensional Gaussian distribution.

The paper is organized as follows. The used data-set and the computational
methods are described in detail in Sec. \ref{sec:methods}. In Sec.
\ref{sec:results} the results for the both logistic regression- and
Bayes-classifier are given. Sec. \ref{sec:conclusions} is a
summary of the results and the differences between the two classifiers are
discussed.

\section{Used data-set and computational methods}
\label{sec:methods}
\subsection{Used data-set}
\label{sec:used_data_set}
The data-set consisted of 4363 songs and was divided into training and test data
sets including every third song to test set and rest of the songs to training
set. Each song contained 264 features and the songs were labeled to 10
different categories. The gatecories were: 1 Pop Rock, 2 Electronic, 3 Rap,
4 jazz, 5 Latin, 6 RnB, 7 International, 8 Country, 9 Reggae and 10 Blues.
The musical characteristics of the songs were packed to a feature
vector of length 256. The first 48 elements in the feature vector can be
associated to timbre, the next 48 elements to pitch and the final 168 features
to rhythm. The distribution of the features resembled in most cases a Gaussian
distribution or a skew symmetric distribution. This is illustrated figures
\ref{fig:feature_distribution}a and \ref{fig:feature_distribution}b.


\subsection{Computational methods}
\label{sec:computational_methods}
In this work two different methods were used to classify the songs to different
genres. First method is logistic-regression method in which the logistic-loss
is minimized iteratively using the gradient descent method.
The other method used is the Bayes-classifier which classifies the song to
certain category that gives the maximum posterior probability with respect to
label $i$. Both methods are described below in detail. In addition we studied
the effect of feature extraction and for that purpose we used principal
component analysis method to exclude features with little impact. 
\subsubsection{Logistic-regression}

\subsubsection{Principal component analysis}

\begin{center}
  \begin{table}
    \caption{Confusion matrix corresponding to classification obtained using
      logistic regression. The column direction indicates the true value and
      the row direction is the predicted value.  The labels $1\ldots10$ are
      the ten music genres specified in section \ref{sec:used_data_set}.}
    \begin{tabular*}{0.45\textwidth}{@{\extracolsep{\fill}}c|cccccccccc}
        & 1 & 2 & 3 & 4 & 5 & 6 & 7 & 8 & 9 & 10\\
      \hline
      1 & 652 & 35 & 7 & 9 & 4 & 10 & 1 & 3 & 2 & 3\\
      2 & 57 & 130 & 9 & 4 & 2 & 1 & 0 & 2 & 2 & 0\\
      3 & 14 & 9 & 82 & 4 & 2 & 1 & 0 & 0 & 1 & 0 \\
      4 & 25 & 3 & 0 & 43 & 0 & 2 & 0 & 1 & 0 & 2 \\
      5 & 38 & 4 & 1 & 5 & 11 & 1 & 1 & 0 & 3 & 0 \\
      6 & 37 & 6 & 12 & 8 & 4 & 25 & 0 & 0 & 0 & 0 \\
      7 & 34 & 6 & 1 & 2 & 2 & 4 & 3 & 1 & 0 & 0 \\
      8 & 50 & 0 & 0 & 1 & 2 & 1 & 0 & 9 & 0 & 0 \\
      9 & 2 & 5 & 0 & 2 & 1 & 0 & 0 & 8 & 0 & 27 \\
      10 & 21 & 0 & 0 & 6 & 1 & 2 & 0 & 1 & 0 & 3\\
      \end{tabular*}
    \label{tab:confusion_log_reg}
  \end{table}
\end{center}


\section{Conclusions}
\label{sec:conclusions}
In this work we used logistic-regression and Bayes-classifier to classify songs
to different genres based on the music signal's characteristics. For logistic
regression the obtained accuracy for test set was 0.67 and logistic-loss 0.27.
For the external data set used the obtained accuracy and logistic-loss were 0.65
and 0.178 respectively in the case of logistic-regression. For the
Bayes-classifier the obtained accuracy and logistic-loss were 0.53 and 0.33 for
the test-data and for external data set 0.32 and 1.17 respectively. According to
obtained results both classifiers performed clearly better than random guess,
but remained far from perfect classification. From the two classifiers used
the logistic-regression classifier performed clearly better. The logistic
classifier also generalized much better to completely new data giving nearly
equal performance for test data set and external data set.

\bibliography{references}

\end{document} 






